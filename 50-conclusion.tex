\chapter{ЗАКЛЮЧЕНИЕ}


В результате проделанной работы был разработан необходимые инструменты были проанализированы и разработаны для поддержки лучшего управления отзывами пользователей, чтобы обеспечить лучшее управление между жителями района и жителями с городскими властями.


Хотя для обоснования полезности этих инструментов необходимо использовать их пользователями, ясно, что люди предпочитают сотрудничать друг с другом, когда это оправдывает его (лучше жить в сообществе).
Точно так же взаимосвязь между экземплярами системы позволит обменяться отзывами между кварталами и городами по желанию или просто взять лучшие примеры в каждой области.


Хорошее использование обратной связи, в дополнение к полезности, чтобы дать лучшую рекомендацию, помогает улучшить целые процессы, рекомендовать действия или обеспечить лучшее решение проблем. Кроме того, пользовательская обратная связь, когда используется для управления районами или городами, полезна для того, чтобы сделать правительство прозрачным, а граждане больше всего выигрывают.

С использованием приложений, библиотек и программных сред для управления большими объемами данных также возможно, что в будущем он может быть использован для других функций, которые не были обнаружены до сих пор.


Задачи выполнены полностью, а именно:
\begin{itemize}
\item Была разработана полевая работа по изучению существующих компьютерных средств, способствующих развитию городских анклавов.

\item Были проанализированы существующие потребности кварталов, а инструменты были разработаны для поддержки городских анклавов, когда возникла необходимость в улучшении предложения садовых продуктов или домашней краски или путем добавления предложений культурной или развлекательной деятельности при сравнении более развитых районов, чем другие;

\itemТочно так же был изучен способ разработки инструмента поддержки совместной экономики, то есть обмена отзывами пользователя или тех, кто заинтересован в присоединении к этим методологиям работы;

\itemОн был разработан и очерчен инструментом, который поддерживает тип «системы инцидентов», чтобы жители могли информировать правительство о проблемах, которые происходят в окрестностях, чтобы иметь решение;

\itemРазработан и набросал инструмент, который отражает идеи жителей, чтобы улучшить анклавы и что они могут внести свой вклад в план, голосовать за них и представлять правительству;

\itemОни научились использовать некоторые инструменты, которые поддерживают управление BigData, чтобы иметь возможность выполнять указанные ранее и в то же время быть готовыми к новым требованиям;

\end{itemize}