\chapter{ТЕХНОЛОГИЧЕСКИЙ РАЗДЕЛ}

\section{Масштабируемость}


Масштабируемость системы, помимо прочего, определяется способностью системы управлять большим объемом данных. Сначала это может показаться не проблемой, но приложения, в основном розничные или розничные, с большим количеством структурированных или неструктурированных данных, имеют тенденцию к экспоненциальному росту.


С этим мы ссылаемся на так называемые BigData \cite{bigdata}, которые относятся к наборам данных или комбинациям наборов данных, размер, сложность которых и скорость роста которых затрудняют сбор, управление, обработку или анализ с использованием традиционных технологий и инструментов, таких как как реляционные базы данных и обычные статистические данные, такие как база данных Mysql \cite{mysql} с использованием языка SQL \cite{sql}, в течение приемлемого времени, поскольку нет смысла иметь ответ через 5 минут.


Поскольку мы не определили, какое количество данных определяет BIGDATA, мы будем искать лучшее решение с наименьшими затратами и обеспечивающее масштабируемость, что, в свою очередь, позволяет нам взаимодействовать с внешними инструментами, такими как ERP, CRM или ad-hoc-системы.

Преимущества BIGDATA можно найти в обширной биографии на эту тему, но поскольку это ключевой момент в рассматриваемой системе, мы подробно описываем:

\begin{itemize}
	\item Быстро для принятия решений Используя соответствующий инструмент и способный объединять разные источники данных, можно сразу анализировать информацию и принимать решения на основе того, что они узнали.


\item Снижение стоимости при использовании сторонних сервисов на основе облака. Хранение большого количества данных в конкретных службах помогает определить более эффективные способы ведения бизнеса.

\item Новые услуги По мере появления новых потребностей с готовностью bigdata предоставление решений становится проще.
\end{itemize}

Теперь ищите лучшее решение для управления большими объемами данных, и что оно доступно и с открытым исходным кодом.

\section{Выберите подходящее вам решение}

Характеристика системы, как мы упоминали в разделе анализа, представляет собой приложение на основе сервисов, которое может быть принято как экземпляр, который распределяется между другими экземплярами.

Первым вариантом является Apache Hadoop, который представляет собой программную среду, которая поддерживает приложения, распространяемые по бесплатной лицензии. Hadoop позволяет приложениям работать с тысячами узлов и большими объемами данных, имеет относительно большое сообщество пользователей и работает под Java.

Второй альтернативой является Apeche Spark, которая идет немного дальше, потому что это кластерное решение для вычислений и обеспечивает интерфейс для программирования полных кластеров с параллелизмом данных и высокой отказоустойчивостью.

Apache Spark - это все-в-кластерное, ориентированное на скорость решение, которое также предоставляет api для Python, R, Java и Scala. Он также имеет встроенную библиотеку для машинного обучения. Теперь сравните Hadoop и Apache Spark:

\begin{enumerate}
	\item \textbf{трудность:} В то время как Apache Spark имеют много операторов высокого уровня, которые делают программирование и поддержку RDD с Hadoop, разработчики должны кодировать каждую операцию, которая делает работу трудно.
	\item \textbf{Простота использования:} Apache Spark может работать в пакетном, интерактивном режиме, режиме машинного обучения или потоковой передачи в одном кластере. Это целый механизм анализа данных, поэтому нет необходимости использовать разные компоненты для каждой потребности. Единая установка Apache Spark охватывает все потребности.
	\item \textbf{Скорость отклика:} Apache Spark - это громоздкий инструмент для кластерных вычислений и запускает приложения со скоростью 100 раз быстрее в памяти и десять быстрее на диске, чем Hadoop. Из-за уменьшения числа циклов R-W на диск и хранения промежуточных данных в памяти Spark позволяет.
Hadoop читает и записывает с диска, с низкой производительностью и низкой скоростью обработки.
	\item \textbf{Отказоустойчивость:} Apache Spark является отказоустойчивым. В результате нет необходимости перезапускать приложение с нуля в случае сбоя. Hadoop MapReduce, как Apache Spark, MapReduce также отказоустойчив, поэтому нет необходимости перезапускать приложение с нуля в случае сбоя.
	\item \textbf{Безопасность:} Apache Spark немного менее безопасен по сравнению с Hadoop, поскольку он поддерживает единственную аутентификацию через аутентификацию с общим секретным паролем. Hadoop MapReduce более безопасен из-за Kerberos, а также поддерживает списки контроля доступа (ACL), которые являются традиционной моделью разрешения файлов.
\end{enumerate}

После этого сравнения ясно, что Apache Spark намного удобнее, если нас не беспокоит сильная безопасность, поэтому мы выбираем Spark, поскольку у него много преимуществ по сравнению с Hadoop\cite{comparison}.


\subsection{Основные модули Spark}

В нашем случае мы попробуем только первый, но посмотрим, что они собой представляют:

Spark SQL - это модуль Spark для обработки структурированных данных. Он обеспечивает абстракцию программирования, называемую DataFrame, а также может выступать в роли распределенного механизма запросов SQL. DataFrame - это распределенный сбор данных, организованный в столбцах.
Концептуально он эквивалентен таблице в реляционной базе данных, но оптимизирован.

GraphX ​​- это новый компонент Spark для графиков и параллельного вычисления графиков. На высоком уровне GraphX ​​расширяет Spark RDD, введя новую абстракцию Графа: прямой мультиграф со свойствами, связанными с каждой вершиной и ссылкой.

Spark Streaming - это расширение ядра API Spark, которое обеспечивает масштабируемую, отказоустойчивую и высокопроизводительную поточную обработку. Он обеспечивает абстракцию высокого уровня, называемую Dstream, которая представляет непрерывное прибытие данных. Внутренне Dstream представлен как последовательность RDD.

Mllib - это масштабируемая библиотека обучения компьютера Spark, состоящая из общих алгоритмов обучения и утилит, включая классификацию, регрессию, кластеризацию, совместную фильтрацию, уменьшение размерности.

