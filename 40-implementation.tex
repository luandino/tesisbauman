\chapter{ТЕХНОЛОГИЧЕСКИЙ РАЗДЕЛ}

\section{Модуль ядра}

Сам модуль ядра написана на языке С с использованием встроенного в ОС Linux компилятора GCC. Выбор языка основан на том, что исходный код ядра, предоставляемый системой, написан на С,  и использование другого языка программирования в данном случае было бы нецелесообразным. 

\begin{lstlisting}[language=C, caption={Функция, обрабатывающая запись в специальный файл},label=module_dev_write]
int module_dev_write(struct file *sp_file,const char __user *buf, size_t size, loff_t *offset) {
	printk(KERN_INFO "module_dev: called write %d\n",size);
	
	copy_from_user(msg, buf, size);
	length_msg = size;

	is_empty = 0;

	int event;

	if (msg[0] == '1') {
		event = 1;
  		tasklet_schedule(&my_tasklet);
	} else {
		event = 0;
	}

	printk(KERN_INFO "module_dev: write event = %d\n", event);

	return size;
}
\end{lstlisting}

\section{Работа с индикаторами клавиатуры}

Доступ к индикаторам клавиатуры реализован через виртуальную консоль, доступную в ядре через библиотеку linux/console\_struct.h.

Отложенное действие реализовано через тасклет. 

Для исключения возможности одновременного выполнения двух копий одной и той же программы в ядре используется mutex.

\begin{lstlisting}[language=C, caption={Поставновка таймера на выполнение},label=lab1]
static DECLARE_MUTEX(mutex);


void my_tasklet_function(void) {

	down(mutex);

	int i;
    printk(KERN_INFO "kbleds: loading\n");
    printk(KERN_INFO "kbleds: fgconsole is %x\n", fg_console);

    for (i = 0; i < MAX_NR_CONSOLES; i++) {
            if (!vc_cons[i].d)
                    break;
            printk(KERN_INFO "poet_atkm: console[%i/%i] #%i, tty %lx\n", i,
                   MAX_NR_CONSOLES, vc_cons[i].d->vc_num,
                   (unsigned long)vc_cons[i].d->port.tty);
    }

    printk(KERN_INFO "kbleds: finished scanning consoles\n");
    my_driver = vc_cons[fg_console].d->port.tty->driver;
    printk(KERN_INFO "kbleds: tty driver magic %x\n", my_driver->magic);
    
    /*
     * Set up the LED blink timer the first time
     */
    init_timer(&my_timer);
    my_timer.function = my_timer_func;
    my_timer.data = (unsigned long)&my_timer_func_data;
    my_timer.expires = jiffies + START_DELAY;
    add_timer(&my_timer);

    return;
}
\end{lstlisting}

\begin{lstlisting}[language=C, caption={Функция таймера, включающая/выключающая индикаторы},label=lab2]
#define SHORT_DELAY   HZ / 5
#define LONG_DELAY   3 * HZ / 5
#define START_DELAY   HZ

#define ALL_LEDS_ON   0x07
#define RESTORE_LEDS  0xFF

int sos[9] = {SHORT_DELAY, SHORT_DELAY, SHORT_DELAY, LONG_DELAY, LONG_DELAY, LONG_DELAY, SHORT_DELAY, SHORT_DELAY, SHORT_DELAY};
int pos = 0;

static void my_timer_func(unsigned long ptr) {
	int *pstatus = (int *)ptr;
	
	if (pos == 9) {
		pos = 0;
		del_timer(&my_timer);
	    (my_driver->ops->ioctl) (vc_cons[fg_console].d->port.tty, KDSETLED, RESTORE_LEDS);
		up(mutex);
	    return;
	}

	printk(KERN_INFO "kbleds: pstatus = %d\n", *pstatus);

	int delay;

    if (*pstatus == ALL_LEDS_ON) {
        *pstatus = RESTORE_LEDS;
		delay = SHORT_DELAY;
    } else {
		*pstatus = ALL_LEDS_ON;
		delay = sos[pos];

		++pos;
    }

    (my_driver->ops->ioctl) (vc_cons[fg_console].d->port.tty, KDSETLED, *pstatus);

    my_timer.expires = jiffies + delay;

    add_timer(&my_timer);
}
\end{lstlisting}


\section{Программа пользователя}

Программа пользователя также написана на языке С, т.к. ALSA предоставляет свой API для этого языка.

\begin{lstlisting}[language=C, caption={Подготовка к считыванию входа микрофона},label=PrepareCapture]
#define ALSA_PCM_NEW_HW_PARAMS_API

#include <alsa/asoundlib.h>

  /* Open PCM device for recording (capture). */
  rc = snd_pcm_open(&handle, "default", SND_PCM_STREAM_CAPTURE, 0);
  
  if (rc < 0) {
    fprintf(stderr,
            "unable to open pcm device: %s\n",
            snd_strerror(rc));
    exit(1);
  }

  /* Allocate a hardware parameters object. */
  snd_pcm_hw_params_alloca(&params);

  /* Fill it in with default values. */
  snd_pcm_hw_params_any(handle, params);

  /* Set the desired hardware parameters. */

  /* Interleaved mode */
  snd_pcm_hw_params_set_access(handle, params,
                      SND_PCM_ACCESS_RW_INTERLEAVED);

  /* Signed 16-bit little-endian format */
  snd_pcm_hw_params_set_format(handle, params,
                              SND_PCM_FORMAT_S16_LE);

  /* Two channels (stereo) */
  snd_pcm_hw_params_set_channels(handle, params, 1);
  // snd_pcm_hw_params_set_channels(handle, params, 2);

  /* 44100 bits/second sampling rate (CD quality) */
  val = 44100;
  snd_pcm_hw_params_set_rate_near(handle, params, &val, &dir);

  /* Set period size to 32 frames. */
  frames = 32;
  snd_pcm_hw_params_set_period_size_near(handle, params, &frames, &dir);

  /* Write the parameters to the driver */
  rc = snd_pcm_hw_params(handle, params);

  if (rc < 0) {
    fprintf(stderr,
            "unable to set hw parameters: %s\n",
            snd_strerror(rc));
    exit(1);
  }

  /* Use a buffer large enough to hold one period */
  snd_pcm_hw_params_get_period_size(params, &frames, &dir);

  /* 2 bytes/sample, 2 channels */
  size = frames * 2; 
  
  buffer = (char *) malloc(size);

  snd_pcm_hw_params_get_period_time(params, &val, &dir);
\end{lstlisting}

\begin{lstlisting}[language=C, caption={Обнаружение нажатия},label=PrepareCapture]
	MIN_NECESSARY_MARKS = 8;
	flag = 0;

	while (1) {
    	rc = snd_pcm_readi(handle, buffer, frames);

    	for (i = 0; i <  size - 1; ++i) {
    	  if (buffer[i] == 0x7f && buffer[i + 1] == (char)0xff) {
    	    ++flag;
    	    ++i;
    	    
    	    if (flag > MIN_NECESSARY_MARKS) {
    	        system("echo 1 > /proc/module_dir/dev");
    	        flag = 0;
    	    }
    	  } else {
    	        flag = 0;
    	  }
    	}
    }

\end{lstlisting}
