\Introduction

Прежде всего, я благодарен за предоставленную возможность учиться в университете Баумана в Москве, особенно учителей  для получения образования, и за большое терпение у них с иностранными студентами, другим властям Университета, сотрудники деканата и другие чиновники, которые сделали и способствовали моему опыту, были лучшими. Я также хотел бы поблагодарить Министерство образования России за предоставленную возможность и российскую общину в целом за то, что у меня была возможность встретиться.

В настоящее время разработка и \textbf{Рекомендальная Система} или «RS» - это многодисциплинарные усилия, которые выиграли от результатов, полученных в различных областях компьютерных наук, в особенности машинного обучения и интеллектуального анализа данных, поиска информации и взаимодействия человека и компьютера. Между машинным обучением и интеллектуальным анализом данных существуют четкие отношения как подполя искусственного интеллекта, которые позволяют компьютеру научиться оптимально выполнять определенную задачу с использованием примеров, данных или прошлого опыта.

Например, «Data mining» \cite{data_mining}можно использовать для изучения данных транзакций у клиентов, которые купили «джинсы», также купили «футболки». Следовательно, рекомендации могут быть построены с использованием информации, предоставленной этими ассоциациями. Актуальные темы сосредоточены на использовании различных алгоритмов машинного обучения и «интеллектуального анализа данных» для прогнозирования пользовательских оценок для элементов или для обучения правильному ранжированию элементов для пользователя.

Двумя наиболее важными типами «RS» являются: «контент-основанный» и «комбинированная фильтрация». Рекомендатор «Content-based», который пытается порекомендовать элементы, подобные тем, которые были у других пользователей, понравился в прошлом, тогда как «Colaborative filtering» - это метод автоматического прогнозирования (фильтрации) интересов пользователя путем сбора предпочтений или информации о вкусах от многие пользователи (сотрудничающие). Основополагающее предположение подхода «Коллаборативная фильтрация» заключается в том, что если лицо «А» имеет такое же мнение, как лицо «В» по эмиссии, «А» с большей вероятностью имеет мнение «Б» по другой эмиссии, случайно выбранного человека.

В другой стороне отдельных систем рекомендаций может случиться так, что рекомендации могут иметь отношение к группе пользователей, а не к индивидууму \cite{porsuit}. 



В рекомендательной системе, ориентированной на отдельных лиц, нас интересует только максимальная индивидуальная удовлетворенность, и для этого достаточно всегда рекомендовать элемент с наивысшим рейтингом. Таким образом, нет необходимости точно прогнозировать удовлетворение. Однако, если мы заинтересованы в поддержании удовлетворенности группы, то внезапно становится более важно точно предсказать индивидуальное удовлетворение.

Чтобы остальная часть группы была счастлива, человеку, возможно, придется иногда сталкиваться с предметами, которые им не нравятся. Поэтому важно знать, не выбрали ли выбранные предметы не слишком неудовлетворенность отдельных лиц. Точные прогнозы индивидуальной удовлетворенности также могут помочь оценить стратегии адаптации групп.


В настоящее время индустрия сосредоточена на разработке надежных алгоритмов, которые могут дать надежные рекомендации, более сложные для злонамеренных пользователей, а также о том, что в алгоритмах используется масштабируемость для работы в больших наборах данных. Фактически, системы совместных рекомендаций зависят от доброй воли своих пользователей с неявным состоянием. И наоборот, академики ищут проблемы, которые могут быть решены в рамках научного сообщества. Это сделало и затруднит отраслевое и академическое сотрудничество.

Исследования выиграли от совокупного интереса и усилий, которые промышленность и научные круги имеют, но таким же образом требуют новых конкретных задач, и существует риск застоя, если мы не сможем решить полезные, но рискованные задачи в пользу проблем.

Основываясь на теории «разбитых окон» \cite{brokenw}, которая предполагает, что видимые признаки пренебрежения и плохого обслуживания в условиях окружающей среды поощряют преступность и беспорядок, в том числе серьезные преступления. Предложение начинается с прототипа программного обеспечения, разрабатывает модель, которая может быть использована в качестве развития кварталов, основанная на этих фактах: при разработке соседства существуют другие районы, уже разработанные в определенной области, которые содержат полезные данные, которые могут использоваться как рекомендации (видимые улучшения дома, культурные предложения и события и эволюция окрестностей). Если можно дать рекомендации для соседей, примените их к своим домам и рекреационной жизни, в результате общая окружающая среда получит качественное обновление. Пара проектов \cite{oregon}\cite{nextdoor} намерена развивать окрестности, но без поддержки RS до сих пор.


\textbf{Постановка задачи} Будет разработана рекомендательная система  для выбора продуктов для дома и рекреационных или культурных мероприятий, людьми, которые живут в определенном районе.  Исследование дополняется исследованиями и разработкой инструмента, в котором использование информации из системы рекомендаций по потребительским привычкам и возможное включение экспертных предложений может предоставлять неавтоматические рекомендации с целью предоставления преимуществ для как в обществе, в котором они живут.

Оценка этих рекомендаций исходит из сопоставления городского анклава, который хочет спровоцировать позитивные изменения, с другим городским анклавом, который был квалифицирован как удовлетворительный, хороший или превосходный в сравниваемом предмете.

В основном в двух областях вам необходимо получить рекомендации: рекомендации по приобретению товаров и / или мероприятий по благоустройству дома и рекомендации по рекреационной деятельности. На оба они, возможно, повлияли на добавление политик, установленных экспертами в каждой области.


Для достижения этой цели необходимо решить следующие задачи:


\begin{itemize}
\item Проведите анализ соответствующего метода фильтрации рекомендаций для каждой группы; необходимо будет проанализировать альтернативы алгоритмов, сравнивая их с соответствующими метриками;
\item В случае действий или рекомендаций, которые включают в себя ряд этапов, которые должны быть завершены, определите подходящий метод, который будет использоваться, оценивая доступные альтернативы;
\item Когда в рекомендациях, касающихся деятельности, участвует группа людей, будет определен способ установления удовлетворенности групп соответствующими методами;
\item Включение механизм логического вывода для определения рекомендаций для развития городских анклавов. Необходимые правила для включения рекомендаций начинаются с установленного уровня удовлетворенности, первоначально оцениваемого, но который может быть скорректирован с учетом обратной связи с окружающей средой и знаний экспертов в каждой области;
\item Чтобы оценить положительную оценку конкретного района, активность и удовлетворенность его жителей в отношении определенного Установить в эмпирическом масштабе, что считается развитым рядом в предметах, подлежащих изучению. Если это необходимо, сообщите другим заинтересованным сторонам, о квалификации, достигнутой развитыми секторами, и определите план, который потребуется для его достижения (действий);
\item Включить модуль для уведомления о рекомендациях в автономном режиме, когда необходимо отправить их пользователям соответствующей системы;
\item Убедитесь, что разработанные методы дают приемлемые решения;

\end{itemize}

\textbf{\textit{"Цель этой работы заключается в том, что из понимания математических и алгоритмических оснований модели «один к одному» или «поставщик-клиент», типичной для системы рекомендаций, разработан инструмент, который включает такие методы для приобретения товаров, но с добавлением других, которые сотрудничают в общем улучшении среды, в которой живут эти пользователи."}}










